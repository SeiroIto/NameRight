\begin{longtable}{>{\hfill\footnotesize }p{3cm}<{}>{\hfill\footnotesize }p{3cm}<{}>{\hfil\footnotesize }p{2cm}<{}>{\footnotesize }p{0.75cm}<{}>{\footnotesize }p{5cm}<{}}
\rowcolor{lightblue}\makebox[3cm]{case} & \makebox[3cm]{ref} & \makebox[2cm]{state} & \makebox[0.75cm]{year} & \makebox[5cm]{remark}\\ \endfirsthead
\rowcolor{lightblue}\makebox[3cm]{case} & \makebox[3cm]{ref} & \makebox[2cm]{state} & \makebox[0.75cm]{year} & \makebox[5cm]{remark}\\ \endhead
Stuart v. Bd. of Supervisors of Elections for Howard County & 295 A.2d 223 & Maryland & 1972 & \mpage{5cm}{\footnotesize Holding that marriage did not, as a matter of law, change the wife's surname to that of the husband.\hfill\setlength{\baselineskip}{10pt}}\\\rowcolor{gray90}
Custer v. Bonadies & 318 A.2d 639 Conn. Super. Ct. & Connecticut & 1974 & \mpage{5cm}{\footnotesize Finding that neither common law nor statute compels a married woman to take her husband's surname, although it is the custom.\hfill\setlength{\baselineskip}{10pt}}\\
Marshall v. State & 301 So. 2d 477 Fla. Dist. Ct. App. & Florida & 1974 & \mpage{5cm}{\footnotesize Allowing a married woman to claim her nonmarital name as her legal name.\hfill\setlength{\baselineskip}{10pt}}\\\rowcolor{gray90}
In re Hauptly & 312 N.E.2d 857 & Indiana & 1974 & \mpage{5cm}{\footnotesize Holding that a married woman has the same right to change her name as anyone else.\hfill\setlength{\baselineskip}{10pt}}\\
MAss. Op. ATr'y GEN. & Number 5 at 48 & Massachusetts & 1974 & \mpage{5cm}{\footnotesize Finding that Massachusetts law does not compel a woman retaining her maiden name after marriage to assume her husband's surname for any purpose.\hfill\setlength{\baselineskip}{10pt}}\\\rowcolor{gray90}
In re Halligan & 361 N.Y.S.2d 458 App. Div. & New York & 1974 & \mpage{5cm}{\footnotesize Holding that the potential confusion which might arise when a woman bore a name different from her husband's was an insufficient reason to deny her application for judicial name change.\hfill\setlength{\baselineskip}{10pt}}\\
In re Natale & 527 S.W.2d 402, 404-05 Mo. Ct. App. & Montana & 1975 & \mpage{5cm}{\footnotesize Noting that restricting a woman's right to use the name of her choice is inconsistent with developments granting women equal legal rights.\hfill\setlength{\baselineskip}{10pt}}\\\rowcolor{gray90}
In re Lawrence & 337 A.2d 49, 51 N.J. Super. Ct. App. Div. & New Jersey & 1975 & \mpage{5cm}{\footnotesize Finding a "woman may retain her maiden name by antenuptial agreement or by holding herself out consistently by that name after marriage."\hfill\setlength{\baselineskip}{10pt}}\\
In re Mohlman & 216 S.E.2d 147 N.C. Ct. App. & North Carolina & 1975 & \mpage{5cm}{\footnotesize Stating that at marriage a woman does not give up her right to change her name as anyone else might change his or hers.\hfill\setlength{\baselineskip}{10pt}}\\\rowcolor{gray90}
In re Strikwerda & 220 S.E.2d 245 & Verginia & 1975 & \mpage{5cm}{\footnotesize Finding that nothing in the wording of statute purports to exclude a married woman from petitioning the court to change her name from her married name to her maiden name.\hfill\setlength{\baselineskip}{10pt}}\\
Kruzel v. Podell & 226 N.W.2d 458,459 & Wisconsin & 1975 & \mpage{5cm}{\footnotesize Finding error in an election board's purging of voter registration where a woman did not take her husband's surname at marriage.\hfill\setlength{\baselineskip}{10pt}}\\\rowcolor{gray90}
Weathers v. Superior Court of Los Angeles & 126 Cal. Rptr. 547 Ct. App. & California & 1976 & \mpage{5cm}{\footnotesize Allowing a married woman to sue for divorce in her own name.\hfill\setlength{\baselineskip}{10pt}}\\
FLA. ATr'y GEN. ANN. REP. & \textsection 076-66  at 120 & Florida & 1976 & \mpage{5cm}{\footnotesize Advising that for purposes of voter registration, the "true" name of a married woman who chooses to retain her birth surname is her given name and her birth surname, not her given name and her husband's surname.\hfill\setlength{\baselineskip}{10pt}}\\\rowcolor{gray90}
Brown v. Brown & 384 A.2d 632, 632 & District of Columbia & 1977 & \mpage{5cm}{\footnotesize Finding no limitation in the common law for "any adult or emancipated person" to change his or her name at will.\hfill\setlength{\baselineskip}{10pt}}\\
Secretary of the Commonwealth v. City Clerk of Lowell & 366 N.E.2d 717 & Massachusetts & 1977 & \mpage{5cm}{\footnotesize Recognizing that a woman may change her name at will, without resort to legal proceedings.\hfill\setlength{\baselineskip}{10pt}}\\\rowcolor{gray90}
Ball v. Brown & 450 F. Supp. 4 N.D. & Ohio & 1977 & \mpage{5cm}{\footnotesize Finding error in an election board's purging of voter registration where a woman did not take her husband's surname at marriage.\hfill\setlength{\baselineskip}{10pt}}\\
ME. Op. ATr'Y GEN. & WL 33940 Me.A.G. & Maine & 1978 & \mpage{5cm}{\footnotesize Ruling that both women and men have option of retaining their surnames after marriage.\hfill\setlength{\baselineskip}{10pt}}\\\rowcolor{gray90}
Simmons v. O'Brien & 272 N.W.2d 273 & Nebraska & 1978 & \mpage{5cm}{\footnotesize Finding error in lower court's refusal to grant a divorce in wife's separate surname because common law did not compel married woman to bear the same surname as her husband.\hfill\setlength{\baselineskip}{10pt}}\\
In re Miller & 243 S.E.2d 464 & Verginia & 1978 & \mpage{5cm}{\footnotesize Noting that no statute in Virginia requires a married woman to assume her husband's surname, despite custom.\hfill\setlength{\baselineskip}{10pt}}\\\rowcolor{gray90}
Traugott v. Petit & 404 A.2d 77 & Rhode Island & 1979 & \mpage{5cm}{\footnotesize Upholding the common law right of a divorced woman to use the name of her choice.\hfill\setlength{\baselineskip}{10pt}}\\
Malone v. Sullivan & 605 P.2d 447 & Arizona & 1980 & \mpage{5cm}{\footnotesize Finding error in a trial court's refusal to entertain a woman's divorce petition unless she amended her pleading to reflect her surname as her husband's.\hfill\setlength{\baselineskip}{10pt}}\\\rowcolor{gray90}
State v. Taylor & 415 So. 2d 1043, 1047 & Alabama & 1982 & \mpage{5cm}{\footnotesize Finding that "in view of the fact that the common law regarding 'names' has not been altered by the legislature." Alabama adopts the common law of England that a woman's change of name upon marriage is in fact rather than in law.\hfill\setlength{\baselineskip}{10pt}}\\
\end{longtable}
